\chapter{<Conclusions>}
Describe the conclusions (reflect on the whole introduction given in Chapter 1).

Discuss the positive effects and the drawbacks.

Describe the evaluation of the results of the degree project.

Describe valid future work.

The sections below are optional but could be added here.


Conclusion ist dass es nicht geeignet ist fuer ieine anwendung aber es schon ein system regeln kann.\\
Dann in final words sagen was wir gemacht haben und iwie auf das intro gehen
\section{Discussion}

Answer the questions of the problem!!!!\\

\subsection{Future Work}
Es so machen dass es waehrend dem task noch lernt. Z.b mit EWC oder dem Zenke ansatz fuer snn\\
Maybe different noise models.. Brown noise
Adjust the input such that the imbalance can be negated and training is faster.\\
find nonlinearity\\
end point control\\
make the network acting as controller learn as well.\\
Test if learning with lower amplitude towards the end can refine the performance even more.\\
\todo{Double rate adjustment. In epoch and overall epochs}

Further work:\\
Maybe learning methods to control nonlinear dynamic systems\\
Maybe we can even do the adverserial attack to try to screw with the network.\\
Implement this on neuromorphic hardware\\

\subsection{Final Words}

\listoftodos






