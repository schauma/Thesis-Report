\newpage
\thispagestyle{plain}
~\\
\vfill
{ \setstretch{1.1}
	\subsection*{Author}
	Max Schaufelberger <maxscha@kth.se>\\
	School of Engineering Sciences\\
	KTH Royal Institute of Technology

	\subsection*{Place for Project}
	Stockholm, Sweden\\
	Ottignies-Louvain-la-Neuve, Belgium

	\subsection*{Examiner }
	Prof. Elias Jarlebring\\
	Department of Numerical Analysis\\
	KTH Royal Institute of Technology\\
	Stockholm, Sweden
	~
	\subsection*{Supervisor }
	Arvind Kumar\\
	Division of Computational Science and Technology\\
	KTH Royal Institute of Technology\\
	Stockholm, Sweden
	~
	\subsection*{Supervisor }
	Frédéric Crevecoeur\\
	Institute of Information and Communication Technologies, Electronics and Applied Mathematics\\
	UCLouvain Catholic University of Louvain\\
	Louvain-la-Neuve, Belgium
	~


}


\newpage
\thispagestyle{plain}
%%%%%%%%%%%%%%%%%%%%%%%%%%%%%%%%%%%%
%%  The English abstract          %%
%%%%%%%%%%%%%%%%%%%%%%%%%%%%%%%%%%%%
\chapter*{Abstract}
%%%%%%%%%%%%%%%%%%%%%%%%%%%%%%%%%%%%

%This is a template for writing thesis reports for the ICT school at KTH. I do not own any of the images provided in the template and this can only be used to submit thesis work for KTH.

The emergence of spiking networks as the third generation of neural networks has shown great success in solving various tasks. Here, networks of spiking neurons are used to control a linear system using biologically plausible methods. Spiking neurons are introduced, and different frameworks highlighted. The control concept consists of two efficient coding networks: one for generating the necessary input to drive the second network, simulating the dynamics. The precise network behaviour is explained using a geometric methods. Network parameters for the simulating network are learned using supervised and unsupervised learning rules.\\
For the simulation, results from the spiking network are accurate for various system sizes.\\
Applying the control network to the dynamic system directly, independent of the simulating network, yields valid results as long as conditions on the input and output matrices are met.\\
Acceptable results for control using two networks can be reached if either the learning or the input matrix of the problem is neglected.\\
Control using learned matrices is limited by inaccuracies in the supervised learning of matrix parameters as well as problem-dependent tuning of hyper-parameters. Moreover, learning progress is hard to monitor without repeated testing by simulation.\\
The results of this thesis suggest that the methods are unable to capture a general black box problem of designing a controller but can be useful when additional information is available.


\subsection*{Keywords}
Spiking networks, Control, Efficient Coding Networks, Learning



\newpage
\thispagestyle{plain}
%%%%%%%%%%%%%%%%%%%%%%%%%%%%%%%%%%%%
%%	 The Swedish abstract         %%
%%%%%%%%%%%%%%%%%%%%%%%%%%%%%%%%%%%%
\chapter*{Abstract}
%%%%%%%%%%%%%%%%%%%%%%%%%%%%%%%%%%%%
Framväxten av spikande nätverk som den tredje generationen av neurala nätverk har visat sig vara mycket framgångsrik när det gäller att lösa olika uppgifter. Här används nätverk av spikande neuroner för att styra ett linjärt system med hjälp av biologiskt trovärdiga metoder. Spikande neuroner introduceras och olika ramverk belyses. Kontrollkonceptet består av två effektiva kodningsnätverk: ett för att generera den input som krävs för att driva det andra nätverket, som simulerar dynamiken. Det exakta nätverksbeteendet förklaras med hjälp av geometriska metoder. Nätverksparametrarna för det simulerande nätverket lärs in med hjälp av regler för övervakad och oövervakad inlärning.
För simuleringen är resultaten från spiking-nätverket exakta för olika systemstorlekar.\\
Att tillämpa kontrollnätverket direkt på det dynamiska systemet, oberoende av simuleringsnätverket, ger giltiga resultat så länge villkoren för in- och utgångsmatriserna är uppfyllda.\\
Acceptabla resultat för styrning med två nätverk kan uppnås om antingen inlärnings- eller indatamatrisen för problemet försummas.\\
Styrning med hjälp av inlärda matriser begränsas av felaktigheter i den övervakade inlärningen av matrisparametrar samt problemberoende inställning av hyperparametrar. Dessutom är det svårt att övervaka inlärningsförloppet utan upprepade tester genom simulering.
Resultaten i denna avhandling tyder på att metoderna inte kan fånga ett generellt black box-problem med att utforma en regulator men kan vara användbara när ytterligare information finns tillgänglig.

\subsection*{Nyckelord}
Spikande nätverk, kontroll, Efficient Coding networks, lärande

\newpage
\thispagestyle{plain}
\chapter*{Acknowledgements}


I would like to express my sincere gratitude to my supervisors, Arvind Kumar and Elias Jarlebring, for their invaluable guidance, support, and encouragement throughout the research and writing of this thesis. Their expertise and constructive feedback have been instrumental in shaping the quality and direction of this work.

I am deeply appreciative of the time and effort they devoted to providing insightful suggestions and helping me navigate the challenges of this project.

In conclusion, I extend my sincere appreciation to Arvind Kumar and Elias Jarlebring for their unwavering support and mentorship. Their guidance has been instrumental in making this research endeavour a rewarding and successful experience.

\newpage

\chapter*{Acronyms}

\begin{acronym}[RDBMS]

\acro{DS}{Dynamic System}
\acrodefplural{DS}{Dynamic Systems}

\acro{NN}{Neural Network}
\acrodefplural{NN}{Neural Networks}

\acro{ANN} {Artificial Neural Network}
\acrodefplural{ANN}{Artificial Neural Networks}

\acro{SNN}{Spiking neural network}
\acrodefplural{SNN}{Spiking neural networks}


\acro{ACID}{atomicity, consistency, isolation, and durability}
\acro{CAP}{Consistency, Availability, Partition-tolerant}
\acro{CDF}{Cumulative Distribution Function}
\acro{CPU}{Central Processing Unit}
\acro{IF}{Integrate and Fire}
\acro{LIF}{Leaky-integrate-and-fire}
\acro{RHS}{Right Hand Side}
\acro{HH}{Hodgkin–Huxley}
\acro{NLP}{Natural Language Processing}
\end{acronym}




\newpage

\etocdepthtag.toc{mtchapter}
\etocsettagdepth{mtchapter}{subsection}
\etocsettagdepth{mtappendix}{none}
\thispagestyle{plain}
\tableofcontents

\newpage


