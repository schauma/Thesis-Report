\newpage
\thispagestyle{plain}
~\\
\vfill
{ \setstretch{1.1}
	\subsection*{Author}
	Max Schaufelberger <maxscha@kth.se>\\
	School of Engineering Sciences\\
	KTH Royal Institute of Technology

	\subsection*{Place for Project}
	Stockholm, Sweden\\
	Ottignies-Louvain-la-Neuve, Belgium

	\subsection*{Examiner }
	Prof. Elias Jarlebring\\
	Department of Numerical Analysis\\
	KTH Royal Institute of Technology\\
	Stockholm, Sweden
	~
	\subsection*{Supervisor }
	Arvind Kumar\\
	Division of Computational Science and Technology\\
	KTH Royal Institute of Technology\\
	Stockholm, Sweden
	~
	\subsection*{Supervisor }
	Frédéric Crevecoeur\\
	Institute of Information and Communication Technologies, Electronics and Applied Mathematics\\
	UCLouvain Catholic University of Louvain\\
	Louvain-la-Neuve, Belgium
	~


}


\newpage
\thispagestyle{plain}
%%%%%%%%%%%%%%%%%%%%%%%%%%%%%%%%%%%%
%%  The English abstract          %%
%%%%%%%%%%%%%%%%%%%%%%%%%%%%%%%%%%%%
\chapter*{Abstract}
%%%%%%%%%%%%%%%%%%%%%%%%%%%%%%%%%%%%

%This is a template for writing thesis reports for the ICT school at KTH. I do not own any of the images provided in the template and this can only be used to submit thesis work for KTH.

The emergence of spiking networks as the third generation of neural networks has shown great success in solving various tasks. Here, networks of spiking neurons are used in order to control a linear system using biologically plausible methods. Spiking neurons are introduced and different frameworks highlighted. The control concept consist of 2 Efficient coding networks, one for generating the necessary input to drive the of the second network, simulating the dynamics. Network parameters for the simulating network are learned using supervised and unsupervised learning rules.\\
Results explain a visual and geometric interpretation of the network dynamics. Acceptable results for control using 2 networks can be reached if either the learning or the input matrix of the problem are neglected. In addition, a parameter study towards the learning rules is delivered.\\
Control using learned matrices is limited by inaccuracies of the supervised learning of matrix parameters as well as problem dependent tuning of hyper-parameters, prohibiting the use of this approach as a usable spiking network controller.


\subsection*{Keywords}
Spiking networks, Control, Efficient Coding networks, Learning





\newpage
\thispagestyle{plain}
%%%%%%%%%%%%%%%%%%%%%%%%%%%%%%%%%%%%
%%	 The Swedish abstract         %%
%%%%%%%%%%%%%%%%%%%%%%%%%%%%%%%%%%%%
\chapter*{Abstract}
%%%%%%%%%%%%%%%%%%%%%%%%%%%%%%%%%%%%
Spiking-nätverk, som är den tredje generationen av neurala nätverk, har visat sig vara mycket framgångsrika när det gäller att lösa olika uppgifter. Här används nätverk av spikande neuroner för att styra ett linjärt system med hjälp av biologiskt trovärdiga metoder. Spikande neuroner introduceras och olika ramverk belyses. Kontrollkonceptet består av 2 effektiva kodningsnätverk, ett för att generera nödvändig input för att driva det andra nätverket, som simulerar dynamiken. Nätverksparametrar för det simulerande nätverket lärs in med hjälp av regler för övervakad och oövervakad inlärning.
Resultaten förklarar en visuell och geometrisk tolkning av nätverksdynamiken. Acceptabla resultat för styrning med 2 nätverk kan uppnås om antingen inlärnings- eller indatamatrisen för problemet försummas. Dessutom levereras en parameterstudie mot inlärningsreglerna.\\
Styrning med hjälp av inlärda matriser begränsas av felaktigheter i den övervakade inlärningen av matrisparametrar samt problemberoende inställning av hyperparametrar, vilket förhindrar användningen av denna metod som en användbar spiknätverksstyrenhet.


\subsection*{Nyckelord}
Spikande nätverk, kontroll, Efficient Coding networks, lärande

\newpage
\thispagestyle{plain}
\chapter*{Acknowledgements}


I would like to express my sincere gratitude to my supervisors, Arvind Kumar and Elias Jarlebring, for their invaluable guidance, support, and encouragement throughout the research and writing of this thesis. Their expertise and constructive feedback have been instrumental in shaping the quality and direction of this work.

I am deeply appreciative of the time and effort they devoted to providing insightful suggestions and helping me navigate the challenges of this project.

In conclusion, I extend my sincere appreciation to Arvind Kumar and Elias Jarlebring for their unwavering support and mentorship. Their guidance has been instrumental in making this research endeavour a rewarding and successful experience.

\newpage

\chapter*{Acronyms}

\begin{acronym}[RDBMS]

\acro{AD}{Automatic Differentiation}

\acro{ML}{Machine Learning}

\acro{RL}{Reinforcement Learning}

\acro{DS}{Dynamic System}
\acrodefplural{DS}{Dynamic Systems}

\acro{LTI}{Linear Time Invariant}

\acro{RNN}{Recurrent Neural Network}
\acrodefplural{RNN}{Recurrent Neural Networks}

\acro{NN}{Neural Network}
\acrodefplural{NN}{Neural Networks}

\acro{GD}{Gradient Descent}

\acro{EWC}{Elastic Weight Consolidation}

\acro{ANN} {Artificial Neural Network}
\acrodefplural{ANN}{Artificial Neural Networks}

\acro{SNN}{Spiking neural network}
\acrodefplural{SNN}{Spiking neural networks}

\acro{GPU}{Graphics Processing Unit}
\acrodefplural{GPU}{Graphics Processing Units}

\acro{BPTT}{Backpropagation Through Time}
\acro{RTRL}{Real Time Recurrent Learning}

\acro{SRDP}{Spike Rate Dependent Plasticity}
\acro{STDP}{Spike Time Dependent Plasticity}

\acro{LSM}{Liquid State Machine}
\acrodefplural{LSM}{Liquid State Machines}

\acro{NEF}{Neural Engineering Framework}

\acro{SOP}{Synaptic Operation}
\acro{ACID}{atomicity, consistency, isolation, and durability}
\acro{CAP}{Consistency, Availability, Partition-tolerant}
\acro{CDF}{Cumulative Distribution Function}
\acro{CPU}{Central Processing Unit}
\acro{IF}{Integrate and Fire}
\acro{LIF}{Leaky-integrate-and-fire}
\acro{TTFS}{Time to First Spike}


\acro{ODE}{ordinary differential equation}
\acro{LHS}{Left Hand Side}
\acro{RHS}{Right Hand Side}
\acro{HH}{Hodgkin–Huxley}
\acro{NLP}{Natural Language Processing}
\acro{LQG}{Linear Quadratic Gaussian}
\acro{LQR}{Linear Quadratic Regulator}
\end{acronym}




\newpage

\etocdepthtag.toc{mtchapter}
\etocsettagdepth{mtchapter}{subsection}
\etocsettagdepth{mtappendix}{none}
\thispagestyle{plain}
\tableofcontents

\newpage


