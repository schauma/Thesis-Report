\chapter{<Theoretical Background>}
% \thispagestyle{fancy}
In this chapter, a detailed description about background of the degree project is presented together with related work. Discuss what is found useful and what is less useful. Use valid arguments.

Explain what and how prior work / prior research will be applied on or used in the degree project /work (described in this thesis). Explain why and what is not used in the degree project and give valid reasons for rejecting the work/research.

Use references!

\section{Use headings to break the text}
Do not use subtitles after each other without text in between the sections.

\section{Related Work}
You should probably keep a heading about the related work here even though the entire chapter basically only contains related work.

Here just what has been done for each of the headlines\\


Neural networks in general
spiking neural networks and their differences and what they are better for.
neuron models, iwazishi neuron and maybe one more
mein neuron model und warum ich es ausgewaelt habe: einfach zu implementieren. Bereits fuer dynamische systeme verwendet,
Nachteile dieses modells.
Vlt vergleich mit einem anderen modell.
Ganz kurzer ausflug in die regelung von dynamischen systemen.


What is a neural network? -> not here ref a paper. kurze erkl'rung in der einfuerung
in der einfuhurng vlt auch hodgekin huxley erwaehen :)
\subsection{Dynamic systems}

\subsection{Neural Networks}

\subsection{Spiking Neural Networks}
A spiking Neural network is one step closer to a biologic representation of a brain. Instead of conveying information using a gradient in conventional \ac{NN}s, information is propagated using discrete spikes of excitation, similar to biological neurons. Hereby one can distinguish between several ideas of implementation.


\subsubsection{Neuron types}
Classic\\
Izhekevich neuron\\

\subsubsection{Poisson-Networks}
TestTTTTT
\subsubsection{Liquid state machines}

\subsubsection{GLM}

\subsubsection{Balanced Networks}

\subsubsection{Learning: SGD and STDP}
Here explain the conpects for each of the NNs\\
Give references for the STDP variances\\
test\\
