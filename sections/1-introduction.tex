\chapter{Introduction}

Provide a general introduction to the area for the degree project. Use references!

Link things together with references. This is a reference to a section: \ref{sec:background}.

\rewrite{Neuro stuff, very rapid development, tremendous progress, many things are successful with NNs. Then list fields that work well. E.g pattern recognition, bioinformatics, neuroscience. With spiking neural networks they are behind the state of the art feedforward networks but the gap is closing. There are already fields where they are excel compared over normal NN.}

Spiking neural networks mimic the inner workings of the brain.

\section{Background}
\label{sec:background}
Present the background for the area. Give the context by explaining the parts that are needed to understand the degree project and thesis. (Still, keep in mind that this is an introductory part, which does not require too detailed description).

Use references\footnote{You can also add footnotes if you want to clarify the content on the same page.}

Detailed description of the area should be moved to Chapter 2, where detailed information about background is given together with related work.

This background presents background to writing a report in latex.


Example citation \cite{Jones2017} or for two authors: \cite{Jones2017, Liu2017}

Look at sample table \ref{tab:sample-table-label} for a table sample.

\begin{table}[!ht]
\centering
\caption{Sample table. Make sure the column with adds up to 0.94 for a nice look.}
~\\
\label{tab:sample-table-label}
\begin{tabular}{p{0.3\textwidth} p{0.64\textwidth}}
\toprule
\textbf{SAMPLE}		  & \textbf{TABLE}                                                                                                                                                  \\ \toprule
One                   & Stuff 1 \\
\midrule
Two                   & Stuff 2 \\
\midrule
Three                 & Stuff 3\\
\bottomrule
\end{tabular}
\end{table}



Boxes can be used to organize content

\begin{tcolorbox}[title={Development environment for prototype}]
	\tt{
		\textbf{Operating systems }\\
		computer: Linux - kernel 4.18.5-arch1-1-ARCH\\
		android phone: 8.1.0\\
		~\\
		\textbf{Build tools}\\
		exp (build tool): version 55.0.4\\
		~\\
		...
	}
\end{tcolorbox}

\section{Problem}
NN have excelled at many fields\\
Fields where they are not fit\\ aka temporal data\\
They have ways to compromise on that \\ -> reference\\
Spiking nn inherently temporal \\
more natural choice\\
However they also have problems\\
like the following:::: reference!!\\
\section{Purpose}
The purpose of the degree project/thesis is the purpose of the written material, i.e., the thesis. The thesis presents the work / discusses / illustrates and so on.

It is not “The project is about” even though this can be included in the purpose. If so, state the purpose of the project after purpose of the thesis).

Probably delete as a own paragraph but mention smth like that.

\section{Goal}
The goal means the goal of the degree project. Present following: the goal(s), deliverables and results of the project.\\
Goal is to write a SNN that can deliver good performance in solving a task that sucks with a conventional NN.
We test the performance of a SNN and a conventional NN for linear dynamic systems.\\
The performance of of the SNN should be close to equal to conventional control schemes and better than conventional NNs.\\
Ideally the SNN has desired features like small number of spikes, precision, learning, poisson distribution etc. more find references.\\
Ideally we stress the network by removing many neurons and see the performance. maybe to recreate the performance of the one paper.. \\
Potentially find the optimal minimal network in the network. The paper dass der Typ aus louvain vorgestellt hat.
\section{Benefits, Ethics and Sustainability}
Describe who will benefit from the degree project, the ethical issues (what ethical problems can arise) and the sustainability aspects of the project.

Use references!

\section{Methodology}
Introduce, theoretically, the methodologies and methods that can be used in a project and, then, select and introduce the methodologies and methods that are used in the degree project. Must be described on the level that is enough to understand the contents of the thesis.

Use references!

Preferably, the philosophical assumptions, research methods, and research approaches are presented here. Write quantitative / qualitative, deductive / inductive / abductive. Start with theory about methods, choose the methods that are used in the thesis and apply.


Detailed description of these methodologies and methods should be presented in Chapter 3. In chapter 3, the focus could be research strategies, data collection, data analysis, and quality assurance.


We build a SNN for a control problem and check it for performance as mentioned above. In addition we design a conventional controller and compare the result. IF we have the time for it we put a conventional NN to it too. We see the performance compared to the others and look at the specs we mentioned above.
The SNN is trained by learning using STDP rule. We can compare the learned weights with the optimal weights when we have our own optimal controller/ we simulate our trajectory.
For our approach we use a balanced spiking network.
\section{Stakeholders}
Present the stakeholders for the degree project.

\section{Delimitations}
Explain the delimitations. These are all the things that could affect the study if they were examined and included in the degree project.
Use references!

\section{Outline}
In text, describe what is presented in Chapters 2 and forward. Exclude the first chapter and references as well as appendix.
