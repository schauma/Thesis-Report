\chapter{Introduction}

Provide a general introduction to the area for the degree project. Use references!

Link things together with references. This is a reference to a section: \ref{sec:background}.

\rewrite{Neuro stuff, very rapid development, tremendous progress, many things are successful with NNs. Then list fields that work well. E.g pattern recognition, bioinformatics, neuroscience. With spiking neural networks they are behind the state of the art feedforward networks but the gap is closing. There are already fields where they are excel compared over normal NN.}


The human brain is a brilliant computing unit comprised of around 86 billion\cite{azevedo_equal_2009} neurons. Each of these neurons can have thousands of connections to other neurons.\addref{Maybe put some exact numbers here and a source}. Between these connections, information travels trough the network as electrical impulses that interact with the neurons own electrical potential. The huge network complex of the human brain is capable of vastly different and intricate tasks.\rewrite{Sounds vague} Some problems that are still next to impossible to solve by machines and classical algorithms alone. Moreover many machine implementations lack the speed, precision or flexibility of the human counterpart.\\
Researchers want to remedy this by mimicking the brain's internal network structure to solve problems deemed unsuitable for classic algorithms.\\
These \acp{ANN} have made impressive progress in the fields of image recognition and \addref{Find some more fields with source!}.



Spiking neural networks mimic the inner workings of the brain. They are often dubbed the third generation of \acp{NN}. On the contrary the more traditional

Human brain amazing\\
We struggle to make to replicate at the computer level with algorithms\\
Neural network a way to imitate the brain and its inner workings.\\
Neural networks have shown and proven performance in certain areas\\
They lack in some others\\
Therefore considerable research went into it\\
Now we have spiking neural networks, that imitate the brain even more\\
We hope that with that we have even better performance\\
From very biological to very abstract there have been many proposals\\
Cost performance trade off.\\
Spiking networks have gained similar or exceeding performance compared to the artificial one in some areas-> refs\\
Key advantage is in the temporal dimensional gain.\\
One field they are suited well is the control of dynamic systems\\
In this thesis we use a spiking neural network to control a linear dynamic system\\
The usual way to simulate biological dynamic systems is using LQG control -> ref\\
I believe because of the energy minimization\\
So we can compare them with usual NN and control in terms of performance... i guess\\
We start by giving an intro into spiking neural networks\\
Then spiking neural networks for dynamic systems\\
After control theory with SNN and maybe regular LQG control\\
Lately the learnign of SNNs for the control of dynamic system\\

Further work:\\
Maybe learning methods to control nonlinear dynamic systems\\
Maybe we can even do the adverserial attack to try to screw with the network.\\
Implement this on neuromorphic hardware\\

Problem:\\
Problem is that it is unnatural for classic NN to use temporal data.\\
They usually quantize it and make a big input layer -> ref\\
There are recurrent networks but ... they need to have smth bad as well\\
The LQG control is also not great for some reason I need to find\\
There are many spkiking network archetypes like poisson and GLm and balanced\\
Also problem is that for some spiking networks learning rules could be hard to come by.\\
There are many prospects though as for example ......->refs\\
Also usually learning rules smth of an inverse and that the brain does not have or do I believe arvind said\\



Method:\\
We use a SNN to to control any arbitrary DS\\
Balanced networks show a some key motives seen in the brain like poisson distribution and smth else ->ref\\
The SNN is to be trained with a STDP rule\\
Then compared to optimal weights\\
Then investigated about robustness and other things as many before\\
One part of robustness is trying to get the most essential nodes of the snn to function well.\\¸
Then we have the potential to find a classic nn and train it with that ???\\
With that out of the way we can compare the performance of all the methods.\\
Then we could study the usability for biological interpretation.\\
Maybe even train time over performance or smth whatever\\

Goal:\\
The goal is to have a performant SNN that can be taught to control any linear system with robustness for nodes and weights degrading.\\
It should be as good as a conventional NN or LQG controller.\\
It should be trained using a method that can resemble the way our brain learns.\\
Should be minimal (aka no poisson), so fast and small \\
It should be general, so any system can be trained on\\
It should converge, but for the Balanced it has already been proven if I am not mistaken-> check and ref\\


Work:\\
Explain the controller method aka what the math of the controller\\
In method explain the balanced and the derivation\\
In work summarize the implementation\\
Same for the conventional NN\\
Summarize the training method\\
Explain and derive the training method in method\\


Results:\\
To everthign mentioned in method for performance and so on\\
Answer the questions of the problem!!!!\\


\section{Background}
\label{sec:background}
Present the background for the area. Give the context by explaining the parts that are needed to understand the degree project and thesis. (Still, keep in mind that this is an introductory part, which does not require too detailed description).

Use references\footnote{You can also add footnotes if you want to clarify the content on the same page.}

Detailed description of the area should be moved to Chapter 2, where detailed information about background is given together with related work.


This background presents background to writing a report in latex.


Example citation \cite{Jones2017} or for two authors: \cite{Jones2017, Liu2017}

Look at sample table \ref{tab:sample-table-label} for a table sample.

\begin{table}[!ht]
\centering
\caption{Sample table. Make sure the column with adds up to 0.94 for a nice look.}
~\\
\label{tab:sample-table-label}
\begin{tabular}{p{0.3\textwidth} p{0.64\textwidth}}
\toprule
\textbf{SAMPLE}		  & \textbf{TABLE}                                                                                                                                                  \\ \toprule
One                   & Stuff 1 \\
\midrule
Two                   & Stuff 2 \\
\midrule
Three                 & Stuff 3\\
\bottomrule
\end{tabular}
\end{table}



Boxes can be used to organize content

\begin{tcolorbox}[title={Development environment for prototype}]
	\tt{
		\textbf{Operating systems }\\
		computer: Linux - kernel 4.18.5-arch1-1-ARCH\\
		android phone: 8.1.0\\
		~\\
		\textbf{Build tools}\\
		exp (build tool): version 55.0.4\\
		~\\
		...
	}
\end{tcolorbox}

\section{Problem}
NN have excelled at many fields\\
Fields where they are not fit\\ aka temporal data\\
They have ways to compromise on that \\ -> reference\\
Spiking nn inherently temporal \\
more natural choice\\
However they also have problems\\
like the following:::: reference!!\\
\section{Purpose}
The purpose of the degree project/thesis is the purpose of the written material, i.e., the thesis. The thesis presents the work / discusses / illustrates and so on.

It is not “The project is about” even though this can be included in the purpose. If so, state the purpose of the project after purpose of the thesis).

Probably delete as a own paragraph but mention smth like that.

\section{Goal}
The goal means the goal of the degree project. Present following: the goal(s), deliverables and results of the project.\\
Goal is to write a SNN that can deliver good performance in solving a task that sucks with a conventional NN.
We test the performance of a SNN and a conventional NN for linear dynamic systems.\\
The performance of of the SNN should be close to equal to conventional control schemes and better than conventional NNs.\\
Ideally the SNN has desired features like small number of spikes, precision, learning, poisson distribution etc. more find references.\\
Ideally we stress the network by removing many neurons and see the performance. maybe to recreate the performance of the one paper.. \\
Potentially find the optimal minimal network in the network. The paper dass der Typ aus louvain vorgestellt hat.
\section{Benefits, Ethics and Sustainability}
Describe who will benefit from the degree project, the ethical issues (what ethical problems can arise) and the sustainability aspects of the project.

Use references!

\section{Methodology}
Introduce, theoretically, the methodologies and methods that can be used in a project and, then, select and introduce the methodologies and methods that are used in the degree project. Must be described on the level that is enough to understand the contents of the thesis.

Use references!

Preferably, the philosophical assumptions, research methods, and research approaches are presented here. Write quantitative / qualitative, deductive / inductive / abductive. Start with theory about methods, choose the methods that are used in the thesis and apply.


Detailed description of these methodologies and methods should be presented in Chapter 3. In chapter 3, the focus could be research strategies, data collection, data analysis, and quality assurance.


We build a SNN for a control problem and check it for performance as mentioned above. In addition we design a conventional controller and compare the result. IF we have the time for it we put a conventional NN to it too. We see the performance compared to the others and look at the specs we mentioned above.
The SNN is trained by learning using STDP rule. We can compare the learned weights with the optimal weights when we have our own optimal controller/ we simulate our trajectory.
For our approach we use a balanced spiking network.
\section{Stakeholders}
Present the stakeholders for the degree project.

\section{Delimitations}
Explain the delimitations. These are all the things that could affect the study if they were examined and included in the degree project.
Use references!

\section{Outline}
In text, describe what is presented in Chapters 2 and forward. Exclude the first chapter and references as well as appendix.
